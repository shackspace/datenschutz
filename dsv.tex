\documentclass[a4paper]{dinbrief}

\usepackage[T1]{fontenc}
\usepackage[utf8]{inputenc}
\usepackage[ngerman]{babel}
\usepackage{lmodern}

\usepackage{marvosym}
\usepackage{graphicx}
\usepackage{tabularx}
\usepackage{multirow}
\usepackage{url}

\address{
  \begin{tabularx}{\textwidth}{Xr}
    \begin{tabbing}
      shack e.V.\\
      Ulmerstr. 255\\
      70327 Stuttgart\\
      \Telefon \ \= \kill
     % \Telefon \> +49 (0)711 21729823\\
      %\Mobilefone \> +49 (0)171 1854655\\
      \Letter \> \url{kontakt@shackspace.de}\\
      \ComputerMouse \> \url{www.shackspace.de}\\
    \end{tabbing} 
    &
    \multirow{2}{*}{\includegraphics[height=3cm]{logo_shack_brightbg}} \\
  \end{tabularx}
}

%\signature{}
%\place{Stuttgart} 
\date{\ }
%\yourmail{}
%\sign{}
%\makelabels
%\phone{}{}
\centeraddress
\nowindowrules
%\nowindowtics
%\handling{}



\begin{document}
\bottomtext{\footnotesize
  \hrulefill\\
  \begin{tabularx}{\textwidth}{XXX}
      shack e.V. & Amtsgericht Stuttgart & IBAN  DE44430609677016809500\\ 
      Ulmer Straße 255 & VR 720809 & BIC  GENODEM1GLS \\
    70327 Stuttgart & &  Institut GLS Gemeinschaftsbank
  \end{tabularx}
}

\begin{letter}{
	}

\subject{}
\opening{}
NAME, VORNAME, STRASSE NR, PLZ, ORT

wird hiermit auf die Wahrung des Datengeheimnisses nach §5 Bundesdatenschutzgesetz (BDSG) verpflichtet.
Diese Verpflichtung besteht auch nach der Beendigung der Tätigkeit oder Mitgliedschaft im shack e.V. fort.

Ich erkläre mit meiner Unterschrift, dass ich die durch den Vorstand des shack e.V. zugänglich gemachten Daten, die ich zur Unterstützung des Vorstandes benötige, vertraulich behandle, in keiner anderen Weise verwerte als es meine Unterstützungsarbeit erfordert und keinem Dritten zugänglich mache. Der Vorstand kann hierzu im Rahmen der gesetzlichen Möglichkeiten eine Ausnahme zulassen und wird mir dies ausdrücklich in geeigneter und nachvollziehbarer Form für den Einzelfall erklären. Sollte ich mehr Daten erhalten, als für meine Unterstützungsarbeit benötigt, gilt Selbiges.

Mir ist auch bewusst, dass der shack e.V. und die betroffenen Personen bei Verstößen gegen die vorgenannten Pflichten zum Daten- und Geheimnisschutz zur sofortigen Geltendmachung von Schadensersatzansprüchen berechtigt sind.

Verstöße gegen den Datenschutz können mit Geldbußen oder Geld- oder Freiheitsstrafen geahndet werden. Die Bußgeld- und Strafvorschriften des §§ 43 und 44 BDSG (Anlage) habe ich zur Kenntnis genommen. Ich verpflichte mich, sämtliche Handlungen zu unterlassen, welche zur Verwirklichung dieser Straftat führen können. 

Diese Datenschutzerklärung bleibt auch dann gültig, wenn ich den shack e.V. verlasse oder meine Unterstützungsarbeit niederlege, oder mich der Vorstand nicht mehr zur Unterstützung benötigt.

Rechtliche Rahmenbedingungen:
\begin{itemize}
	\item Bundesdatenschutzgesetz §§ 5, 43, 44
	\item Telekommunikationsgesetz § 88
	\item Strafgesetzbuch §§ 202, 202a, 202b, 202c, 206, 303a, 303b
\end{itemize}


Ich bestätige durch meine Unterschrift, dass ich diese Datenschutzerklärung gelesen und verstanden habe.

\vfill

\begin{tabular}{lp{2em}l}
	 \hspace{5cm}   && \hspace{5cm} \\\cline{1-1}\cline{3-3}
	 Ort, Datum     && Unterschrift
\end{tabular}

\closing{}
\end{letter}
\end{document}

